\documentclass[a4paper]{article}
%=================
%     PACKAGES
%=================
\usepackage[english]{babel}
\usepackage[utf8]{inputenc}
\usepackage{amsmath}
\usepackage{amsthm}
\usepackage{graphicx}
\usepackage[colorinlistoftodos]{todonotes}
\usepackage{blindtext}
\usepackage{hyperref}

%=====================
%     ENVIRONMENTS
%=====================
% Defintions
\newtheorem*{definition*}{Definition}
\setlength{\parskip}{0pt}
\setlength\parindent{24pt}
% Examples
\newtheorem*{example*}{Example}
\setlength{\parskip}{0pt}
\setlength\parindent{24pt}
% Notes
\newtheorem*{note*}{Note}
\setlength{\parskip}{0pt}
\setlength\parindent{24pt}
% Remarks
\newtheorem*{remark*}{Remark}
\setlength{\parskip}{0pt}
\setlength\parindent{24pt}
% Warnings
\newtheorem*{warning*}{Warning}
\setlength{\parskip}{0pt}
\setlength\parindent{24pt}

%==================
%     COMMANDS
%==================
%========================================================
% Common words
\newcommand{\reedsb}{\textbf{\textit{reedsb.com}}}
\newcommand{\reedsbenus}{\textbf{\textit{reedsb.com/en-us}}}
\newcommand{\config}{\textbf{\textit{CONFIG}}}
\newcommand{\appjs}{\textbf{\textit{app.js}}}
\newcommand{\appminjs}{\textbf{\textit{app.min.js}}}
\newcommand{\WMemail}{\textit{reedsb.webmaster@gmail.com }}
\newcommand{\GHuser}{\textit{reedsb-webmaster}}
%========================================================
% Abbreviations
\newcommand{\OSE}{\text{Office of Student Engagement }}
%========================================================
% Formatting
\newcommand{\minititle}[1]{
  \begin{center}
    \textbf{#1}
  \end{center}
}
\newcommand{\blocktitle}[1]{ \noindent \textbf{#1.}}
\newcommand{\blocktitleit}[1]{ \noindent \textit{#1}}
\newcommand{\note}{\textbf{Note:}}
\newcommand{\example}{\textbf{Example:}}
%========================================================
% Removing vertical space
\newcommand{\trim}{\vspace{-1em}}
\newcommand{\trimm}{\vspace{-2em}}
\newcommand{\trimmm}{\vspace{-3em}}
%========================================================
% Arrows
\newcommand{\la}{\leftarrow}
\newcommand{\Ra}{\Rightarrow}
\newcommand{\ra}{\rightarrow}
%========================================================



\title{The Comprehensive Webmaster Manual}
\author{Reed College}
\date{Spring 2021}

\begin{document}
\maketitle


%====================================================================
%====================================================================
%\centering
\section{Introduction}

Welcome to the Reed Webmaster Manual. This manual is meant to be a comprehensive reference for the Reed College Student Webmaster position. Specifically, the manual will cover the official job duties in addition to many resources $\&$ guides on how to run the $\reedsb$ website.

%====================================================================

\medskip
\begin{center}
\textbf{The Website}
\end{center}

The primary functions of $\reedsb$ are threefold:
\begin{enumerate}
\item Hosting the signatories and clubs databases for Treasury
\item Hosting Funding Poll (FP) and the Student Body Elections (SE)
\item Serving as the main hub for publicizing student events
\end{enumerate}

%====================================================================

\medskip
\begin{center}
\textbf{Positions}
\end{center}

Note that there is only a small distinction between the positions of Webmaster and Assistant Webmaster. The jobs differ in no way other than that the Webmaster (as opposed to the Assistant) are assigned the responsibility to be the liaison between the Webmasters and any relevant party. So, in other words, the Webmaster is tasked with communicating with everyone else. The pays are comporable and differ very slightly.

%====================================================================

\medskip
\begin{center}
\textbf{Office of Student Engagment}
\end{center}
The Webmasters are actually hired by the Office of Student Engagement (OSE). This initiative was set up in Fall 2020 by Al Chen, then SB President. As such, the Reed College Treasury is not responsible for paying the Webmasters' wages. This was done in lieu of the project OSE seeks to undertake with $\reedsb$ in the future. The project seeks to make $\reedsb$ a hub for student body events by unifying all disjoint calendars into this website, making events more accessible.

%====================================================================
\newpage
\medskip
\begin{center}
\textbf{Duties}
\end{center}
The Webmasters' primary duties are ongoing development, maintenance, and updates to the Student Body’s website, $\reedsb$. The website's features are ones that help support the goals of Senate, Treasury, and OSE. As such, it is possible that you may be asked to help any of these parties with projects they may request so you must listen to their demands. So, the Webmaster is also expected to regularly communicate with Treasury, the Senate Website Liaison, and the OSE (who pays the wages).

\begin{remark*}[Prioritize.]
In the end, there are absolute obligations the Webmaster has to meet. So, the Webmaster has to prioritize their effort. For this reason, you should appropiately take on or reject side projects. While you are certainly not bound to agree to every project automatically, you must make sure to communicate this to the requesting party.
\end{remark*}


%====================================================================

\begin{center}
\textbf{Skills}
\end{center}
The webmasters will learn how to use GoDaddy, Git, cPanel, Google Firebase to administer the website. The actual website is programmed with JavaScript, CSS, and PHP. \newline

%====================================================================
%====================================================================
\section{Getting Started}
To get started, we must start to set up your accounts. The very first thing you must do is access the master e-mail, reedsb.webmaster@gmail.com.

You can find the credentials from the credentials document in the documentation repository on Github (reedsb-webmaster/documentation) As mentioned, you will need to use GoDaddy, Firebase, and Github for the website.

For GoDaddy and Github, the same credentials should be sufficient for access. For Firebase, you must be added as an owner. This will be done in orientation.

%====================================================================
%====================================================================
\section{Website Manual}
For the website development manual, there is an external manual available.

%====================================================================
%====================================================================
\section{Continuity}
To ensure continuity and smooth transition of the position, please follow the following guide.
\begin{enumerate}
  \item Make sure to keep the credentials documents updated. Do not store the password on the public credentials document on Github.
  \item Ensure the next webmasters have the credentials they need. Always give the password to the new Webmasters in a video call. Additionally, add them to Firebase as an Owner.
  \item Bring the documentation repository to their attention and give them this manual.
\end{enumerate}

%====================================================================
\end{document}
