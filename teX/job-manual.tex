\documentclass[a4paper]{article}
%=================
%     PACKAGES
%=================
\usepackage[english]{babel}
\usepackage[utf8]{inputenc}
\usepackage{amsmath}
\usepackage{amsthm}
\usepackage{graphicx}
\usepackage[colorinlistoftodos]{todonotes}
\usepackage{blindtext}
\usepackage{hyperref}

%=====================
%     ENVIRONMENTS
%=====================
% Defintions
\newtheorem*{definition*}{Definition}
\setlength{\parskip}{0pt}
\setlength\parindent{24pt}
% Examples
\newtheorem*{example*}{Example}
\setlength{\parskip}{0pt}
\setlength\parindent{24pt}
% Notes
\newtheorem*{note*}{Note}
\setlength{\parskip}{0pt}
\setlength\parindent{24pt}
% Remarks
\newtheorem*{remark*}{Remark}
\setlength{\parskip}{0pt}
\setlength\parindent{24pt}
% Warnings
\newtheorem*{warning*}{Warning}
\setlength{\parskip}{0pt}
\setlength\parindent{24pt}

%==================
%     COMMANDS
%==================
% Other
\newcommand{\ra}{\rightarrow}
\newcommand{\reedsb}{\textbf{\textit{reedsb.com}}}
\newcommand{\reedsbenus}{\textbf{\textit{reedsb.com/en-us}}}
\newcommand{\note}{\textbf{Note:}}
\newcommand{\example}{\textbf{Example:}}
\newcommand{\config}{\textbf{\textit{CONFIG}}}
\newcommand{\appjs}{\textbf{\textit{app.js}}}
\newcommand{\appminjs}{\textbf{\textit{app.min.js}}}


\title{The Comprehensive Webmaster Manual}
\author{Reed College}
\date{Spring 2021}

\begin{document}
\maketitle


%====================================================================
%====================================================================
%\centering
\section{Introduction}

Welcome to the Reed Webmaster Manual. This manual is meant to be a comprehensive reference for the Reed College Student Webmaster position. Specifically, the manual will cover the official job duties in addition to many resources $\&$ guides on how to run the $\reedsb$ website.

\begin{note*}[Office of Student Engagement]
  The Webmasters are employed by the Office of Student Engagement (OSE). This initiative was set up in Fall 2020 by Al Chen, then SB President. As such, the Reed College Treasury is not responsible for paying the Webmasters' wages. This was done in lieu of the project the \OSE seeks to undertake with $\reedsb$ in the future --- that is to make $\reedsb$ a hub for student body events by unifying all disjoint event calendars into this website in hopes of increasing accessibility.
\end{note*}

%====================================================================
\minititle{The Website}
%====================================================================

The primary functions of $\reedsb$ are threefold. For each function, we also list the relevant party you are expected to communicate with. There will be further elaboration on each function and how/what you are expected to communicate about.
\begin{enumerate}
\item Hosting the signatories and clubs databases for Treasury \\
\blocktitleit{Party: Treasury}
\item Hosting Funding Poll (FP) and the Student Body Elections (SE) \\
\blocktitleit{Party: Treasury $\&$ Senate ($respectively$)}
\item Serving as the main hub for publicizing student events \\
\blocktitleit{Party: \OSE}
\end{enumerate}

%====================================================================
\subsection{Contacts}
%====================================================================

\noindent That being said, we now elaborate more on the contacts previously mentioned and provide a brief overview of what you are expected to communicate.

\begin{enumerate}
  \item \textbf{Treasury:} You could either communicate with the Head Treasurer or the Vice Treasurer. You are expected to communicate with Treasury atleast once a semester to update the backend and give treasurers access to the back-end of the website. Additionally, you atleast inform the Head Treasurer (Vice President) whether Funding Poll is ready to be run on the website again. Keep informing them for the purpose of institutional memory; that is, atleast give them the expectation that it will one day be fixed.
  \item \textbf{Senate:} You will either communicate with the Webmaster Liaison (if assigned) or the chair of Appointments Committee (App Comm) if the liaison is not explicitly assigned instead. Like with funding poll, inform the liaison about the status of Student Body elections and whether they will be run on the website every semester.
  \item \textbf{\OSE:} As of Spring 2021, you will communicate with Janice Yang and Megan Simón. As OSE is technically your employer, you are expected to report any updates or progress regarding the website (for all three functions).
\end{enumerate}

%====================================================================
\subsection{Duties}
%====================================================================

All that being said. The primary duty of the Webmasters is \textbf{to work on delivering, maintaining, and developing the three functions listed above;} all else is secondary. Though the website may not do so currently, the expectation is that the Webmaster will work towards that goal and try to make progress for the next generation of Webmasters to continue.

\begin{note*}[Positions]
  Note that there is only a small distinction between the positions of Webmaster and Assistant Webmaster. The jobs differ in no way other than that the Webmaster (as opposed to the Assistant) are assigned the responsibility to be the liaison between the Webmasters and any relevant party. So, in other words, the Webmaster is tasked with communicating with everyone else. The pays are comparable (if not equivalent) and differ very slightly.
\end{note*}

\noindent So, to summarize, the Webmasters' primary duties are ongoing development, maintenance, and updates to the Student Body’s website, $\reedsb$. The website's features are ones that help support the goals of Senate, Treasury, and the \OSE. \\

\noindent As such, it is possible that you may be asked to help any of these parties with projects they may request so you must listen to their demands. So, the Webmaster is also expected to regularly communicate with the contacts listed above.

\begin{remark*}[Prioritize]
In the end, there are absolute obligations the Webmaster has to meet. So, the Webmaster has to prioritize their effort. For this reason, you should appropiately take on or reject side projects. While you are certainly not bound to agree to every project automatically, you must make sure to communicate this to the requesting party.
\end{remark*}

%====================================================================
\newpage
\section{Resources}
%====================================================================

\subsection{Credentials}

To get started, you must start to set up your accounts. The very first thing you must do is access the master e-mail: reedsb.webmaster@gmail.com. You should have recieved the password for this e-mail in your transition meeting.

\begin{warning*}[Password Security]
  Please do not store or write down any password anywhere online - especially the credentials file on Github (it is a public repository). This is a security risk. You should ideally have both a copy of the password in a physical copy or in your computer keychain, the latter of which is usually sufficient (given you are able to retrieve it again if you forget it).
\end{warning*}

\noindent You should be able to access Github with exactly the same credentials as the G-mail account. For GoDaddy, there is a seperate username and password since the account is managed by the Reed Treasury $(money@reed.edu)$. In the future, try to see if it is possible to migrate from the Reed Treasury e-mail to the primary webmaster e-mail (\WMemail). The only issue is that they are repsonsible for billing. Lastly, you should be able to access Firebase with the account you were added with.\newline

\noindent You can find all the credentials from the credentials document in the documentation repository on Github (\GHuser\textit{/documentation}). Ideally, try to keep those credentials updated.

\subsection{Github}

As mentioned above, there is an associated Github account with the primary webmaster e-mail. The Github contains two primary items: the webstack (code for the website) and all relevant documentation (manuals and credentials). The username for the Github account is \GHuser.

\subsection{Google Drive}

Using the primary e-mail account, you should be able to immediately access Google Drive, which has the following useful files:

\begin{enumerate}
 \item A backup copy of the Webstack
 \item A backup of this documentation'
 \item Google Docs version of this Manual (to be shared with OSE)
\end{enumerate}

\subsection{Website Manual}
In addition to this manual, there is an external website development manual available. You can find it on either Github or Google Drive.

%====================================================================
%====================================================================

\newpage

%====================================================================
%====================================================================
\section{Continuity}
To ensure continuity and smooth transition of the position, please make sure of the following:
\begin{enumerate}
  \item Keep the credentials documents updated and do not store the password on the public credentials document on Github.
  \item Ensure the next webmasters have the credentials they need. Always give the password to the new Webmasters in a video call. Additionally, add them to Firebase as an Owner.
  \item Bring the documentation repository to their attention and give them this manual.
\end{enumerate}

%====================================================================
\end{document}
